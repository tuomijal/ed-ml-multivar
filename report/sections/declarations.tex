\clearpage
\section*{Declarations}

\subsection*{Ethics approval and consent to participate}
Since our study was retrospective in nature, ethics committee approval was not required. An institutional approval was acquired prior to data collection, with the following specifications:
 
\begin{itemize}
    \item Name: Potilaslogistiikan häiriötekijöiden tunnistaminen ja mallintaminen
    \item Number: PSHP/R19565
    \item Date: 16 June 2019
\end{itemize}

\subsection*{Consent for publication}
Not applicable

\subsection*{Availability of data and materials}
The datasets used and/or analysed during the current study are not currently available due to legislative restrictions.

\subsection*{Conflict of interest statement}
NO is a shareholder in Unitary Healthcare Ltd., which has developed a patient logistics system currently used in the study emergency department. JT, AR and EP are shareholders in Aika Analytics Ltd., which is a company specialized in time series forecasting.

\subsection*{Funding}
The study was funded by the Finnish Ministry of Health and Social Welfare via the Medical Research Fund of Kanta-Häme Central Hospital; the Finnish Medical Foundation; the Competitive State Research Financing of the Expert Responsibility Area of Tampere University Hospital, Pirkanmaa Hospital District, Grant 9X040 and Academy of Finland Grant 310617; Hauho Oma Savings Bank Foundation and Renko Oma Savings Bank Foundation.

\subsection*{Authors’ contributions}
Study design (AR, JT, NO, AP, JP, JK). Data collection (JT, NO). Data-analysis (EP, JT). Technical supervision (JK, JP). Medical supervision (AR, NO, AP). Manuscript preparation (AR, JT, EP). All authors have read and approved the final manuscript.

\subsection*{Acknowledgements}
We acknowledge Unitary Healthcare Ltd for providing the dataset on available hospital beds, the City of Tampere for providing timestamps for public events, and Tampere University Hospital information management for providing website visit statistics. We also acknowledge CSC - IT Center for Science for providing computational resources and specifically D.Sc. Mats Sjöberg for technical support.
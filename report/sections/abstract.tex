Emergency department (ED) crowding is a significant threat to patient safety and it has been repeatedly associated with increased mortality. Forecasting future service demand has the potential to improve patient outcomes. Despite active research on the subject, proposed forecasting models have become outdated due to quick influx of advanced machine learning models (ML) and the amount of multivariable input data has been limited. In this study, we document the performance of a set of advanced ML models in forecasting ED occupancy 24 hours ahead. We use electronic health record data from a large, combined ED with an extensive set of explanatory variables, including the availability of beds in catchment area hospitals, traffic data from local observation stations, weather variables and more. We show that DeepAR, N-BEATS, TFT and LightGBM all outperform traditional benchmarks with up to 15\% improvement. The inclusion of the explanatory variables enhances the performance of TFT and DeepAR but fails to significantly improve the performance of LightGBM. To the best of our knowledge, this is the first study to extensively document the superiority of ML over statistical benchmarks in the context of ED forecasting.

\keywords{Emergency department, Crowding, Overcrowding, Forecasting, Multivariable analysis, Occupancy}